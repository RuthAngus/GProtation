\documentclass[useAMS, usenatbib, preprint, 12pt]{aastex}
\usepackage{cite, natbib}
\usepackage{float}
\usepackage{epsfig}
\usepackage{cases}
\usepackage[section]{placeins}
\usepackage{graphicx, subfigure}
\usepackage{color}
\usepackage{bm}
\usepackage{enumerate}

\newcommand{\columbia}{1}
\newcommand{\oxford}{3}
\newcommand{\sagan}{4}
\newcommand{\uw}{5}
\newcommand{\princeton}{2}

\newcommand{\Kepler}{{\it Kepler}}
\newcommand{\kepler}{\Kepler}
\newcommand{\corot}{{\it CoRoT}}
\newcommand{\Ktwo}{{\it K2}}
\newcommand{\ktwo}{\Ktwo}
\newcommand{\TESS}{{\it TESS}}
\newcommand{\LSST}{{\it LSST}}
\newcommand{\Wfirst}{{\it Wfirst}}
\newcommand{\SDSS}{{\it SDSS}}
\newcommand{\PLATO}{{\it PLATO}}
\newcommand{\Teff}{$T_{\mathrm{eff}}$}
\newcommand{\teff}{$T_{\mathrm{eff}}$}
\newcommand{\FeH}{[Fe/H]}
\newcommand{\feh}{[Fe/H]}
\newcommand{\ie}{{\it i.e.}}
\newcommand{\eg}{{\it e.g.}}
\newcommand{\logg}{log \emph{g}}
\newcommand{\dnu}{$\Delta \nu$}
\newcommand{\numax}{$\nu_{\mathrm{max}}$}

\newcommand{\naigrain}{333}
\newcommand{\nmcquillan}{100}
\newcommand{\nkois}{1102}
\newcommand{\nkoimcq}{275}
\newcommand{\kepexample}{5809890}
\newcommand{\kepexampleperiod}{30.5}
\newcommand{\aigrainexampleperiod}{20.8}

\newcommand{\lnacfMAD}{0.04}
\newcommand{\acfMAD}{0.60}
\newcommand{\percentacfMAD}{3.65}
\newcommand{\lnpgramMAD}{0.04}
\newcommand{\pgramMAD}{0.98}
\newcommand{\percentpgramMAD}{4.32}
\newcommand{\lngpMADnp}{0.02}
\newcommand{\gpMADnp}{0.48}
\newcommand{\percentgpMADnp}{2.45}
\newcommand{\lngpMAD}{0.02}
\newcommand{\gpMAD}{0.39}
\newcommand{\percentgpMAD}{1.96}

\newcommand{\dfmcomment}[1]{{\color{red}#1}}
\newcommand{\tdmcomment}[1]{{\color{blue}#1}}
\newcommand{\vrcomment}[1]{{\color{magenta}#1}}
\newcommand{\racomment}[1]{{\color{green}#1}}

\begin{document}

\title{Inferring probabilistic stellar rotation periods using Gaussian
processes}

\author{%
   Ruth Angus\altaffilmark{\columbia},
   Timothy Morton\altaffilmark{\princeton},
   Suzanne Aigrain\altaffilmark{\oxford},
   Daniel Foreman-Mackey\altaffilmark{\uw,\sagan}
   \& Vinesh Rajpaul\altaffilmark{\oxford}
}
\altaffiltext{\columbia}{Simons Fellow, Department of Astronomy, Columbia
University, NY, NY, RuthAngus@gmail.com}
\altaffiltext{\princeton}{Department of Astrophysical Sciences, Princeton
University,  Princeton, NJ}
\altaffiltext{\oxford}{Subdepartment of Astrophysics, University of Oxford,
UK}
\altaffiltext{\uw}{Department of Astronomy, University of
Washington, Seattle, WA}
\altaffiltext{\sagan}{Sagan Fellow}

\begin{abstract}
The light curves of spotted, rotating stars often vary in a non-sinusoidal and
quasi-periodic fashion---spots move on the stellar surface and have finite
lifetimes, causing stellar flux variations to slowly shift in phase.
A strictly periodic sinusoid therefore cannot accurately model a
rotationally modulated stellar light curve.
Physical models of the stellar surface also have many drawbacks preventing
    effective inference, such as highly degenerate or high-dimensional
    parameter spaces.
% \citep[\eg][]{Russell1906, Jeffers2009, Kipping2012}.
In this work, we introduce an appropriate {\it effective} model: a Gaussian
    Process with a quasi-periodic covariance kernel function.
Using this highly flexible model, we are able to sample from the posterior
    Probability Density Function (PDF) of the periodic parameter,
    marginalising over the other kernel hyperparameters using a Markov Chain
    Monte Carlo approach.
To test the effectiveness of this method, we infer rotation periods from
\naigrain\ simulated stellar light curves, demonstrating that our results
    are more accurate than both a sine-fitting periodogram and an
    autocorrelation function method.
We also demonstrate that it works effectively on real data, by
inferring rotation periods for \nkoimcq\ \Kepler\ stars with
previously measured periods.
Because this method delivers posteriors PDFs, it will enable hierarchical
    studies involving stellar rotation, particularly those involving
    population modelling, such as inferring stellar ages, obliquities in
    exoplanet systems, or characterising star-planet interactions.
% Thehe ACF method is not applicable to unevenly spaced data,
% therefore it is inappropriate for the ~$10^7$ stellar light curves expected
% to be produced by \LSST\ unlike the GP method.
% Furthermore, the improvement is expected to be even more dramatic when applied
% to real, noisy {\it Kepler} light curves, since the GP method is well suited
% to modelling rotation signals and correlated noise simultaneously.

\end{abstract}

\section{Introduction}
\label{sec:intro}

The brightness of a spotted, rotating star often varies in a non-sinusoidal and
Quasi-Periodic (QP) manner, due to active regions on its surface which
rotate in and out of view.
Complicated surface spot patterns produce non-sinusoidal variations,
and the finite lifetimes of these active regions and
differential rotation on the stellar surface produce quasi-periodicity
\citep{Dumusque2011}.
A strictly periodic sinusoid is therefore not necessarily a good model choice
for these time-series.
A physically realistic model of the stellar surface
would, ideally,  perfectly capture the complexity of shapes
within stellar light curves as well as the quasi-periodic nature, allowing for
extremely precise probabilistic period recovery when conditioned on the data.
However, such physical models require many free parameters in order to
accurately represent a stellar surface, and some of these parameters are
extremely degenerate \citep[\eg][]{Russell1906, Jeffers2009, Kipping2012}.
In addition to global stellar parameters such as inclination and rotation
period, each spot or active region should have (at minimum) a longitude,
latitude, size, temperature and lifetime.
Considering that stars may have hundreds of spots, the number of free
parameters in such a model quickly becomes unwieldy, especially to explore its
posterior Probability Density Function (PDF).
Simplified spot models, such as the one described in \citet{Lanza2014} where
only two spots are modelled, have produced successful results; however, such
relatively inflexible models sacrifice precision.

Standard methods to measure rotation periods include detecting
peaks in a Lomb-Scargle \citep{Lomb1976, Scargle1982} (LS) periodogram
\citep[\eg][]{Reinhold2013}, Auto-Correlation Functions (ACFs),
\citep[\eg][]{Mcquillan13b} and wavelet transforms \citep[\eg][]{Garcia2014}.
The precisions of the LS periodogram and wavelet methods are limited by the
suitability of the model choice: a sinusoid for the LS periodogram,
and a choice of mother wavelet, assumed to describe the data
over a range of transpositions \citep[see, \eg][]{Carter2010},
for the wavelet method.
In contrast, the ACF method is much better suited to signals that are
non-sinusoidal.
In fact, as long as the signal is approximately periodic the ACF will
display a peak at the rotation period, no matter its shape.
A drawback of the ACF method, however, is that it requires data to be
evenly-spaced\footnote{\citet{Edelson1988} describe a method for computing
ACFs for unevenly-spaced data.}, which is not exactly the case with \Kepler\
light curves (although in many cases it can be approximated as uniformly
sampled) and not nearly the case for most ground-based measurements, such
as the future Large Synoptic Survey Telescope (\LSST).
An ACF is also an operation performed on the data rather than a generative
model of the data, and so is not inherently probabilistic.
This means that the effects of the observational uncertainties cannot be
formally propagated to constraints on the rotation period.
% Many rotation periods in the literature have been inferred by measuring the
% position of the first peak in an ACF, however this approach can be dangerous.
% The exponential decay in correlation can shift the peak position short-wards
% of its true value, leading to an underestimate of the period.
% We return to this point in section \textsection \ref{sec:discussion}.

In this work, we introduce an {\it effective} model for rotationally
modulated stellar light curves
which captures the salient behaviour but is not
physically motivated---although some parameters may indeed be
{\it interpreted} as physical ones.
An ideal effective model should have a small number of non-degenerate parameters
and be flexible enough to perfectly capture non-sinusoidal and QP behaviour.
A Gaussian process (GP) model fulfills these requirements. We thus use a GP
as the generative model at the core of a method to probabilistically
infer accurate and precise stellar rotation periods.  This enables us to
estimate the posterior PDF of the rotation period, thereby producing a
justified estimate of its uncertainty.
% while simultaneously marginalizing over
% parameters which model correlated noise.

GPs are commonly used in the machine learning community and increasingly
in other scientific fields such as biology, geophysics and cosmology.
More recently, GPs have been used in the stellar and exoplanet fields within
astronomy, to capture stellar variability or instrumental systematics
\citep[see \eg][]{Gibson2012, Haywood2014, Dawson2014, Barclay2015,
Haywood2015, Evans2015, Rajpaul2015, Rajpaul2016, Aigrain2016}.
They are useful in regression problems involving any stochastic process,
specifically when the probability distribution for the process is a
multi-variate Gaussian.
If the probability of obtaining a dataset is a Gaussian in $N$ dimensions,
where $N$ is the number of data points, a GP can describe that dataset.
An in-depth introduction to Gaussian processes is provided in
\citet{Rasmussen2005}.

GP models parameterise the covariance between data points by means of a
kernel function.
As a qualitative demonstration, we present the time-series in Figure
\ref{fig:GP_example}: the \kepler\ light curve of KIC \kepexample.
This is a relatively active star that rotates once every $\sim$
\kepexampleperiod\ days, with stochastic variability typical of \kepler\ FGK
stars.
Clearly, data points in this light curve are correlated.
Points close together in time are tightly correlated, and points more
widely separated are loosely correlated.
A GP models this variation in correlation as a function of the separation
between data points; that is, it models the {\it covariance structure} rather
than the data directly.
This lends GPs their flexibility---they can model any time
series with a similar covariance structure.
In addition, a very simple function can usually capture the covariance
structure of a light curve, whereas modelling the time series itself
might require much more complexity.
Figure \ref{fig:GP_example} demonstrates how a GP model fits the light curve of
KIC \kepexample.
% Both provide adequate fits to the data, however only the periodic kernel
% function, `QP' is a useful model because it has a periodic parameter.
% I return to this point shortly.

A range of kernel functions can describe stellar variability.
For example, the most commonly used kernel function, the `Squared
Exponential' (SE), defined as follows, could adequately fit the
KIC \kepexample\ light curve:
\begin{equation}
\label{eq:SE}
k_{i,j} = A \exp \left(-\frac{(x_i - x_j)^2}{2l^2} \right).
\end{equation}
Here $A>0$ is the amplitude of covariance, $l$ is the length scale of
exponential decay, and $x_i-x_j$ is the separation between data points.
The SE kernel function has the advantage of being very simple, with
just two parameters, $A$ and $l$.
If $l$ is large, two data points far apart in $x$ will be tightly correlated,
and if small they will be loosely correlated.
Another property of the SE kernel function is that it produces functions that
are infinitely differentiable, making it possible to model a data set
and its derivatives simultaneously.
% \citep[see \eg][]{Rajpaul2015}
However, The SE kernel function does not well describe the covariance
in stellar light curves, nor is it {\it useful} for the problem of
rotation period inference because it does not capture periodic behaviour.
Inferring rotation periods thus requires a periodic kernel
function.
For this reason, we use the `Quasi-Periodic' kernel.
\citet{Rasmussen2005} model QP variability in CO$_2$ concentration on the
summit of the Mauna Loa volcano in Hawaii \citep[data from][]{Keeling2004}
using a kernel which is the product of a periodic and a SE kernel: the QP
kernel.
This kernel is defined as
\begin{equation}
\label{eq:QP}
k_{i,j} = A \exp \left[-\frac{(x_i - x_j)^2}{2l^2} -
    \Gamma^2 \sin^2\left(\frac{\pi(x_i - x_j)}{P}\right) \right] + \sigma^2
    \delta_{ij}.
\end{equation}
It is the product of the SE kernel function, which describes the overall
covariance decay, and an exponentiated, squared, sinusoidal kernel function
that describes the periodic covariance structure.
$P$ can be interpreted as the rotation period of the star, and $\Gamma$
controls the amplitude of the $\sin^2$ term.
If $\Gamma$ is very large, only points almost exactly one period away are
tightly correlated and points that are slightly more or less than one period
away are very loosely correlated.
If $\Gamma$ is small, points separated by one period are tightly
correlated, and points separated by slightly more or less are still highly
correlated, although less so.
In other words, large values of $\Gamma$ lead to periodic variations with
increasingly complex harmonic content.
This kernel function allows two data points that are separated in time by one
rotation period to be tightly correlated, while also allowing
points separated by half a period to be weakly correlated.
We also use an extra parameter, $\sigma$, which is an additional white noise
term added to the diagonal elements of the covariance matrix.
This can be interpreted to represent underestimation of observational
uncertainties~---~if the uncertainties reported on the data are too small, it
will be non-zero~---~or it can capture any remaining ``jitter'' or residuals
that are not captured by the effective GP model.
We use this QP kernel function to produce the GP model
that fits the \Kepler\ light curve in in Figure \ref{fig:GP_example}.
There are many ways to construct a QP kernel function, involving a range of
choices for both the periodic and a-periodic components of the model.
The kernel function presented above reproduces the behaviour of stellar light
curves but it is not necessarily the only model choice that can do so.
We do not attempt to test any other choices in this paper, noting only that
this kernel function provides an adequate fit to the data.
We leave formal comparisons with other kernel function choices to a future
publication.

To infer a stellar rotation period $P$ from a light curve, we fit this
QP-kernel GP model to the data.  As with any model-fitting exercise, the
likelihood
of the model could be maximised to find the maximum-likelihood value for $P$.
In this study, however, we explore the full posterior PDFs using a Markov
Chain Monte Carlo (MCMC) procedure.  While
this approach comes at a computational cost, such posterior exploration
importantly provides a justified uncertainty estimate.

\begin{figure}
\begin{center}
\includegraphics[width=6in, clip=true]{figures/koi_lc_demo.pdf}
\caption[A light curve with a GP model.]
{Light curve of KIC \kepexample, an active star with a rotation period of
$\sim$ \kepexampleperiod\ days.
The blue line shows a fit to the data using a Gaussian process model with a QP
covariance kernel function.}
\label{fig:GP_example}
\end{center}
\end{figure}

This paper is laid out as follows.
The GP method is described in \textsection \ref{sec:method}.
Its performance is demonstrated and compared with literature methods in
in \textsection \ref{sec:perf_and_comp}.
In \textsection \ref{sec:kepler} we apply our method to real \kepler\ data,
and the results are discussed in \textsection \ref{sec:discussion}.

\section{GP Rotation Period Inference}
\label{sec:method}

In order to recover a stellar rotation period from a light curve using a
quasi-periodic Gaussian process (QP-GP), we sample the following posterior
PDF:
\begin{equation}
\label{eq:posterior}
p({\bm \theta}\,|\,y) \propto \mathcal L(y\,|\,{\bm \theta}) p({\bm \theta}),
\end{equation}
where $y$ are the light curve flux data, $\bm \theta$ are the hyperparameters
of the kernel described in Equation \ref{eq:QP}, $\mathcal L$ is the
QP-GP likelihood function, and $p({\bm \theta})$ is the prior on the
hyperparameters.  Sampling this posterior presents several challenges:
\begin{itemize}
    \item The likelihood evaluation is computationally expensive;
    \item The GP model is very flexible, sometimes at the expense of
    reliable recovery of the period parameter; and
    \item The posterior may often be multimodal.
\end{itemize}
This Section discusses how we address these challenges through
the implementation details of the likelihood (Section \ref{sec:GP_lhood}), priors
(Section \ref{sec:GP_prior}), and sampling method (Section \ref{sec:sampling}).

\subsection{Likelihood}
\label{sec:GP_lhood}

The GP likelihood is similar to the simple Gaussian likelihood
function used for optimisation problems where the uncertainties are
Gaussian and uncorrelated. The latter can be written
\begin{equation}
    \ln \mathcal{L} = -\frac{1}{2}\sum_{n=1}^N\left[\frac{(y_n-\mu)^2}{\sigma_n^2}
    + \ln(2\pi\sigma_n^2)\right],
\end{equation}
\label{eq:chi2}
where $y_n$ are the data, $\mu$ is the mean model and $\sigma_n$ are the
Gaussian uncertainties on the data.
The equivalent equation in matrix notation is
\begin{equation}
\ln \mathcal{L} = -\frac{1}{2}\bf{r}^T\bf{C}^{-1}\bf{r}-\ln|\bf{C}|
    + \frac{N}{2}\log2\pi,
\end{equation}
\label{eq:lhf1}
where $\bf{r}$ is the vector of residuals and $\bf{C}$ is the covariance
matrix,
\begin{eqnarray}
    \mathbf{C} &=& \left (\begin{array}{cccc}
    \sigma^2_1 & \sigma_{2, 1} & \cdots & \sigma_{N, 1} \\
    \sigma_{1, 2} & \sigma^2_2 & \cdots & \sigma_{N, 2} \\
    && \vdots & \\
    \sigma_{1, N} & \sigma_{2, N} & \cdots & \sigma^2_N
\end{array}\right )
\end{eqnarray}
In the case where the uncertainties are uncorrelated, the noise is `white',
(which is a frequent assumption made by astronomers and is sometimes
justified) and the off-diagonal elements of the covariance matrix are zero.
However, in the case where there is evidence for correlated
`noise'\footnote{In our case the `noise' is actually the model!  Incidentally, this approach is the reverse of the regression techniques
usually employed by astronomers.
In most problems in astronomy one tries to infer the parameters that describe
the mean model and, if correlated noise is present, to marginalise over that
noise.
Here, the parameters describing the correlated noise are what we are
interested in and our mean model is simply a straight line at $y=0$.}, as in the
case of \Kepler\ light curves, those off-diagonal elements are non-zero.
With GP regression, a covariance matrix generated by the kernel function
${\bf K}$ replaces ${\bf C}$ in the above equation (for our purposes, the QP
kernel of Equation \ref{eq:QP}).

Evaluating this likelihood for a large number of points can be computationally
expensive.  For example, evaluating
$\mathcal L$ for an entire \Kepler\ lightcurve
($\sim$40,000 points) takes about $\sim$5\,s--- too slow
to perform inference on large numbers of light curves\footnote{All
computational times cited in this section are based on evaluations on a
single core of a 2015 Macbook Pro, 3.1 GHz Intel Core i7.}.
The matrix operations necessary to evaluate the GP likelihood\footnote{These
operations use the fast matrix solver HODLR \citep{Ambikasaran2014},
implemented in the {\tt george} \citep{George} python package.} scale as
$N\ln(N)^2$, where $N$ is the number of data points in the light curve.

We accelerate the likelihood calculation using two complementary strategies:
subsampling the data and splitting the light curve into independent sections.
To subsample \Kepler\ data, for example,
we randomly select 1/30th of the points
in the full light curve (an average of $\sim$1.5 points per day).
This decreases the likelihood evaluation time by a factor of about 50, down to
about 100\,ms.
We then split the light curve into equal-sized chunks containing approximately
300 points per section (corresponding to about 200 days), and evaluate the
log-likelihood as the sum of the log-likelihoods of the individual sections
(all using the same parameters ${\bm \theta}$).
This reduces computation time because the section-based likelihood evaluation
scales as $mn\ln(n)^2$, where $n$ is the number of data points per section and
$m$ is the number of sections.
This method further reduces computation time for a typical light curve
(subsampled by a factor of 30) by about a factor of two, to about 50\,ms.

\subsection{Priors}
\label{sec:GP_prior}

\subsubsection{Non-period Hyperparameters}
\label{sec:nonperiod_prior}

The flexibility of this GP model allows for posterior multimodality and
``over-fitting''--like behavior.
For example, if $l$ is small, the non-periodic factor in the covariance
kernel may dominate, allowing for a good fit to the data without
requiring any periodic covariance structure---even if clear periodic
structure is present.
Additionally, for large values of $\Gamma$ the GP model becomes extremely
flexible and can fit the data without varying the period.
Managing this flexibility to reliably retrieve the period parameter requires
imposing informative priors on the non-period GP parameters.
% While there are no strictly {\it a priori} reasons to believe
% that any of the hyperparameters should take values in any specific range,
% we have found in this work that reasonable priors for any particular
% dataset can be determined by experimentation.
% We note this is a somewhat subjective art, with a goal of
% not allowing the GP to become so flexible that it does not require
% periodicity to fit the data even in the presence of clear periodicity.
In particular, we find it necessary to avoid large values of
$A$ and $\Gamma$, and small values of $l$; though the exact details of what
works well may differ among datasets.
See \textsection \ref{sec:discussion} for a discussion of more physically
motivated priors on the non-period parameters.

% \vrcomment{VR: in my opinion, this is the weakest part of the paper.
% I don't agree that there are no \emph{a priori} reasons to believe that any of
% the hyperparameters should take values in any specific range, and the phrase
% `somewhat subjective art' seems likely to set alarm bells ringing for
% referees/readers.
% Going through the hyperparameters in turn, the following things come to mind.

% \begin{itemize}
% \item Amplitude, $A$: their is a closed form solution for the maximum-likelihood value of $A$ (solve ${\partial _A}\mathcal{L} = 0$). Steve Roberts alluded to this in his 2012 tutorial paper and I know I've worked through the mathematics myself -- offhand I can't remember the solution, but it's a fairly straightforward looking thing involving the precision matrix like $A\sim{{\mathbf{y}}^{\textrm{T}}}{{\mathbf{K}}^{ - 1}}{\mathbf{y}}$ with some scaling constant. I'll see if I can dig up my workings and/or find a reference. I think this means that if you're anyway going to use a flat prior on $A$, your MAP solution will end up corresponding to the ML solution. In which case you could speed things up a good deal by just setting $A$ to its optimal value \emph{ab initio}, and not bothering to explore the full posterior PDF. More importantly though, if for some reason the MAP value you use doesn't correspond more or less to the ML value (e.g.\ if something went wrong with the MCMC convergence) -- in other words, if you end up using a non-optimal value for $A$ -- I've found you can sometimes get rather bizarre behaviour. For example, when I fit data with an obvious period, sometimes I end up recovering a very wrong period if $A$ is also very wrong e.g.\ much too large. Such problems can be avoided quite straightforwardly by constraining $A$ to sensible values.
% \item Harmonic complexity, $\Gamma$: for large values of $l$ ($l\gg P$), $\Gamma$ can be related to the typical number of stationary/non-stationary points of inflection, per period, found in functions drawn from the corresponding GP. This behaviour will be independent of the exact value of $l$. For `intermediate' values of $l$ ($l\gtrsim P$) this can still be done but the behaviour will be sensitive to the value of $l$. And small values of $l$ e.g.\ $l<P$ are probably not relevant -- see below. If useful I could make a plot or two illustrating this. Anyway, the point is that depending on the processes giving rise to observed stellar signals, you may not expect to require certain types of functions to model observed signals. In typical simulations e.g.\ of arbitrary numbers of spots on rotating stars, I think you're unlikely to end up with more than say three or four turning points per period in the disk-integrated photometric signal (as opposed to say $20$; of course $2$ turning points is the minimum for periodic functions). Such simulations, crude though they may be, can be used to come up with physically-motivated priors for $\Gamma$.
% \item Evolution time-scale, $l$: it's sensible to enforce something like $l>P$ if we desire functions that have at least some fairly clear periodicity, as intuitively $l<P$ would mean the functions are allowed to evolve significantly over time scales shorter than one period. So a function $f(t+P)$ would look signficantly different to $f(t)$, in which case any claims of periodicity become dubious. Actually, you can come with a more rigorous criterion (which also takes into account dependencies on $\Gamma$) than this -- see below.
% \item Suzanne and I have shown that provided
% \begin{equation}
% \frac{{{P^2}}}{{{l^2}{\Gamma ^2}}} < \frac{{4\pi }}{3},
% \end{equation}
% the QP covariance function will have at least one inflection point other than at $|x_i-x_j|=0$ (the derivation is straightforward though a bit long). So $\frac{{{P^2}}}{{{l^2}{\Gamma ^2}}} = \frac{{4\pi }}{3}$ corresponds to a critical point at which secondary maxima disappear, and the behaviour of the kernel becomes dominated by the SE term rather than the periodic term. For $\Gamma\sim\mathcal{O}(1)$, this translates into $l\gtrsim P/2$. In practice satisfying this criterion doesn't absolutely guarantee that function draws will have clear evidence of periodicity, but it does give you a quantitative boundary between obviously periodic behaviour and not-so-obviously periodic behaviour (if you're trying to fit rotation periods then obviously you don't want to use an SE-like kernel to do so) and provides some justification for a more stringent criterion such as $l>P$, or $l>2P$ (ensure that functions evolve significantly over time-scales longer than 2 periods), etc.
% \end{itemize}
% The upshot of all of this is that I don't think constraining hyperparameters needs to be some sort of dark art or ad hockery, and it'd do the underlying method a disservice by suggesting as much. As I see it, there are at least two ways to proceed here.
% \begin{enumerate}
% \item Provide as much detail on constraining the hyperparameters in this paper as we possibly can, including quantitative justification, illustrations of qualitatively-different behaviours for different regions of the hyperparameter space, etc. Disadvantages: if we do this it'd be necessary to re-run your fits with the new priors; and the paper would be come a good deal longer, with the focus possibly shifted away from the essential idea of using a GP to infer rotation periods.
% \item Alternatively, at least allude to some of these approaches to constraining hyperparameters (instead of saying that it's a subjective art) but defer a focused discussion on how to do that to a short, separate paper/research note. As I've already done much of the work for such a study, I'd be happy to take the lead on that at some point later this year. Advantage here would be your paper remains more focused and you don't need to re-run any of your fits. Indeed I imagine that if you check your MAP values for $A$, $l$, etc., they'd usually be consistent with any rigorously-motivated priors (since you get generally sensible results). Perhaps we could show in the follow-up paper how using the physically-motivated priors could lead to improved period inference (at least I hope that would be the case!). Any differential rotation tests could also be shifted to the other paper -- see my later remarks on this.
% \end{enumerate}}

\subsubsection{Period}
\label{sec:period_prior}

% While using an uninformed prior on period (e.g., log-flat) often suffices,
% we also develop a method to construct an informed period prior, based on the
% autocorrelation function (ACF).
We use an informed period prior, based on the autocorrelation function (ACF)
of the light curve.
The ACF has proven to be very useful for measuring stellar rotation periods
\citep{Mcquillan2012, Mcquillan13b, Mcquillan2014}; however, the
method has several shortcomings,
most notably the inability to deliver uncertainties, but also
the necessity of several heuristic choices,
such as a timescale on which to smooth the ACF,
how to define a peak, whether the first or second peak
gets selected, and what constitutes a secure detection.
While this paper presents a rotation period inference method
that avoids these shortcomings,
it seems prudent to still use information available from the ACF.
We thus use the ACF to define a prior on period,
which can help the posterior sampling converge on the true period.

We do not attempt to decide which single
peak in the ACF best represents the true rotation period,
but rather we identify several \emph{candidate} periods and define
a weighting scheme in order to create a noncommital, though useful,
multimodal prior.  While this procedure does not avoid heuristic choices,
we soften the potential impact of these choices because we are simply
creating a \emph{prior} for probabilistically justified
inference rather attempting to identify a single correct period.

As another innovation beyond what ACF methods in the literature have
presented, we also bandpass filter the light curves (using a 5th order
Butterworth filter, as implemented in \texttt{SciPy}) before calculating the
autocorrelation.  This suppresses power on timescales shorter than a chosen
minimum period $P_{\rm min}$ and longer than a chosen maximum $P_{\rm max}$,
producing a cleaner autocorrelation signal than an unfiltered light curve.

We use the following procedure to construct a prior for rotation period
given a light curve:

\begin{enumerate}
\item{For each value of $P_i$, where $i = \{1, 3, 5, 10, 30, 50, 100\}$\,d,
we apply a bandpass filter to the light curve using $P_{\rm min}=0.1$\,d
and $P_{\rm max} = P_i$.  We then calculate the ACF of the filtered
light curve out to a maximum lag of $2P_i$ and smooth it with a boxcar
filter of width $P_i/10$.}

\item{We identify the time lag corresponding to the
first peak of each of these ACFs, as well as the first peak's
trough-to-peak height, creating a set of candidate periods
$T_i$ and heights $h_i$.}

\item{We assign a quality metric $Q_i$ to each of these candidate
periods, as follows.  First, we model the ACF as a
damped oscillator with fixed period $T_i$:
\begin{equation}
y = A e^{-t/\tau} \cos{\frac{2\pi t}{T_i} },
\end{equation}
        \racomment{where and $t$ is the lag time, (some missing text here.)}
and find the best-fitting parameters $A_i$ and $\tau_i$ by a non-linear
least squares minimization procedure.  We then define the
following heuristic quality metric:
\begin{equation}
\label{eq:quality}
Q_i = \left(\frac{\tau_i}{T_i}\right) \left(\frac{N_i h_i}{R_i}\right),
\end{equation}
where $h_i$ is the height of the ACF peak at $T_i$,
$N_i$ is the length of the lag vector in the ACF (directly proportional
to the maximum allowed period $P_i$),
and $R_i$ is the sum of squared residuals between the
damped oscillator model and the actual ACF data.  The idea behind this
quality metric is to give a candidate period a higher score if

    \begin{enumerate}[(a)]

    \item{it has many regular sinusoidal peaks, such that the decay
        time $\tau_i$ is long compared to the oscillation period $T_i$,}
    \item{the ACF peak height is high, and}
    \item{the damped oscillator model is a good fit (in a $\chi^2$
      sense) to the ACF, with extra bonus for being
      a good fit over more points (larger $N_i$, or longer $P_i$).}

    \end{enumerate}
}

\item{Given this set of candidate periods $T_i$ and quality metrics $Q_i$,
we finally construct a multimodal prior on the $P$ parameter of the GP
model as a weighted mixture of Gaussians:
\begin{equation}
\label{eq:mixture}
p(\ln P) = \frac {\displaystyle \sum_i Q_i \left(0.9\mathcal N(\ln T_i, 0.2) +
                                          0.05\mathcal N(\ln (T_i/2), 0.2) +
                                          0.05\mathcal N(\ln (2 T_i), 0.2) \right)}
                {\sum_i Q_i}.
\end{equation}
That is, in addition to taking the candidate periods themselves as mixture
components, we also mix in twice and half each candidate period at a lower level,
which compensates for the possibility that the first peak in the ACF may actually
represent half or twice the actual rotation period.  The period width of 0.2 in
log space (corresponding to roughly 20\% uncertainty) is again a heuristic choice,
balancing a healthy specificity with the desire to not have the results of
the inference overly determined by the ACF prior.
}
\end{enumerate}

Incidentally, while we use the procedure described here to create a prior on
$P$ which we use while inferring the parameters of the quasi-periodic GP model,
this same procedure may also be used in the service of a
rotation-period estimating procedure all on its own, perhaps being even more
robust and accurate than the traditional ACF method.  We leave exploration of
this possibility to future work.

\subsection{Sampling}
\label{sec:sampling}

To sample the posterior in a way that is sensitive to potential multimodality,
we use the \texttt{emcee3}\footnote{\url{https://github.com/dfm/emcee3}} MCMC
sampler.
\texttt{emcee3} is the successor to the \texttt{emcee} project
\citep{Foreman-Mackey2013} that includes a suite of ensemble MCMC proposals
that can be combined to efficiently sample more distributions than the stretch
move \citep{Goodman2010} in \texttt{emcee}.
For this project, we use a weighted mixture of three proposals.
First, we include a proposal based on the \texttt{kombine}
package\footnote{\url{https://github.com/bfarr/kombine}} (Farr \& Farr, in
prep.) where a kernel density estimate (KDE) of the density represented by the
complementary ensemble is used as the proposal for the other walkers.
The other two proposals are a ``Differential Evolution (DE) MCMC'' proposal
\citep{terBraak2006, Nelson2014} and the ``snooker'' extension of DE
\citep{terBraak2008}.

We initialise 500 walkers with random samples from the prior and use a
weighted mixture of the KDE, DE, and snooker proposals with weights of 0.4,
0.4, and 0.2 respectively.
We run \vrcomment{50} steps of the sampler at a time, checking for convergence
after each iteration, up to a maximum of \vrcomment{50} iterations.
We declare convergence if the total effective chain length is at least
8$\times$ the maximum autocorrelation time.
When convergence is achieved, we discard the first two autocorrelation lengths
in the chain as a burn-in, and randomly choose 5000 samples as representative
of the posterior.
This fitting process takes several hours for a typical simulated light curve,
though in some cases it can take 12 hours or longer to converge.

\section{Performance and Comparison to Literature: Simulated Data}
\label{sec:perf_and_comp}

In order to benchmark this new rotation period recovery method, we apply
it to a set of simulated light curves
and compare to the performance of established literature methods.
Section \ref{sec:simulations} describes the simulated data we use;
Section \ref{sec:performance} demonstrates the performance of the
QP-GP method; and Section \ref{sec:comparison} compares it to the performance
of the Lomb-Scargle periodogram and autocorrelation function methods.

\subsection{Simulated light curves}
\label{sec:simulations}

We take our test data set from the \citet{Aigrain2015} `hare and hounds'
rotation period recovery experiment.
These light curves result from placing dark, circular spots with slowly
evolving size on the surface of bright, rotating spheres, ignoring
limb-darkening effects.
\citet{Aigrain2015} simulated one thousand such light curves to test the
ability of participating teams to recover both stellar rotation periods
and rotational shear (the amplitude of surface differential rotation).
% Since we are not interested in recovering differential rotation we only used
% those light curves simulated {\it without} differential rotation, of which
% there are \naigraincurves.
However, in this work, in order to focus on demonstrating reliable period recovery,
we select only the \naigrain\ light curves without differential rotation,
as differential rotation may produce additional scatter in the
measured rotation periods.
% In future we intend to test to what extent we can recover
% differential rotation using the GP method and will then use the
% full set of 1000 light curves.

Each of these light curve simulations uses a real \Kepler\ long-cadence time array:
one data point every thirty minutes over a four year duration.
90\% of the rotation periods of the simulations come from a
log-uniform distribution between 10 and 50 days, and 10\% from a log-uniform
distribution between 1 and 10 days.
Figure \ref{fig:period_hist} shows the distribution of solid-body rotation periods.
The simulations also have a range of stellar inclination
angles, activity levels, spot lifetimes and more
(see Table \ref{tab:simulation_parameters}).
In order to preserve \kepler\ noise properties,
\citet{Aigrain2015} add real \kepler\ light curves with no obvious astrophysical
variability to the theoretical rotationally modulated light curves.
Figure \ref{fig:demo_lc} shows an example of a simulated light curve with a
period of \aigrainexampleperiod days.

\begin{table*}
\begin{center}
\caption{Ranges and distributions of parameters used to simulate light curves
in \citet{Aigrain2015}}
\begin{tabular}{lcc}
\hline\hline
    Parameter & Range & Distribution \\
    \hline
    Rotation period, $P_{rot}$ & 10 - 50 days (90\%) & log uniform \\
    & 1 - 10 days (10\%) & log uniform \\
    Activity cycle length & 1 - 10 years & log uniform \\
    Inclination & 0 - 90$^\circ$ & Uniform in $\sin^2i$ \\
    Decay timescale & (1 - 10) $\times P_{rot}$ & log uniform \\
\hline
\end{tabular}
\end{center}
\end{table*}
\label{tab:simulation_parameters}

\begin{figure}
\begin{center}
\includegraphics[width=6in, clip=true]{figures/period_hist.pdf}
\caption{A histogram of the rotation periods used to generate the \naigrain\
simulated light curves in \citet{Aigrain2015}.}
\label{fig:period_hist}
\end{center}
\end{figure}

\begin{figure}
\begin{center}
\includegraphics[width=6in, clip=true]{figures/demo_lc.pdf}
\caption[A simulated light curve.]
{An example simulated light curve. This `star' has a rotation period of
\aigrainexampleperiod\ days.}
\label{fig:demo_lc}
\end{center}
\end{figure}


\subsection{Method Performance}
\label{sec:performance}

We apply the QP-GP inference method described in Section \ref{sec:method}
to each of these 333 simulated light curves.  As discussed in Section
\ref{sec:nonperiod_prior}, reliable inference
requires defining a useful set of priors on the non-period
hyperparameters.
For this simulation dataset, we determined these by first running
the method using very broad priors on all the non-period parameters
(log-flat between -20 and 20)
and then inspecting the distribution of their posteriors for those cases
that successfully recovered the true period.
We also experimented with constraining the allowed ranges of the parameters
after discovering that some regions of parameter space (such as large values
of $A$ and $\Gamma$ and small values of $l$) tended to allow fits that ignored
the desired periodicity.
We list the final priors and bounds this process led us to adopt
in Table \ref{tab:priors}.
These priors were determined experimentally --- see \textsection
\ref{sec:discussion} for a discussion on more physically motivated priors.
% We note that we do not have any specific quantitative justification for the
% use of these priors, except that they produce good results for this particular
% test set of light curves.
% We thus caution that other data sets might have different noise properties,
% potentially requiring modifications to these priors \vrcomment{[VR: might
% want to reword or remove these remarks; see my earlier comments]}
% (e.g., see Section \ref{sec:kepler}).
For the period prior, we tried two different methods: an uninformed (log-flat)
prior between 0.5\,d and 100\,d, and an ACF-informed prior
(Section \ref{sec:period_prior}).

Figures \ref{fig:compare_mcmc_acfprior} and \ref{fig:compare_mcmc_noprior}
summarise our results compared to the injected
`true' stellar rotation periods, for the uninformed and ACF-informed priors on
period, respectively.
To assess the performance of the QP-GP and other period recovery
methods, we compute Median Absolute Deviations (MADs) of the results, relative
to the input periods.
We also compute the Median Relative Deviation (MRD), as a percentage.
These metrics are presented for the three different methods tested in this
paper in table \ref{tab:MADs}.
The informative and uninformative prior versions of the GP method have MRDs of
\percentgpMAD \% and \percentgpMADnp \% respectively.
The marginal posterior distributions of the QP kernel hyperparameters, for the
example simulated light curve in figure \ref{fig:demo_lc}, are shown in
figure \ref{fig:gp_posteriors}.



\begin{table*}
\begin{center}
\caption{Priors and bounds on the natural logarithms of the GP model
    parameters.}
\begin{tabular}{lcc}
Parameter & Prior & Bounds\\
    \hline
    $\ln A$ & $\mathcal N(-13, 5.7)$ & (-20, 0) \\
    $\ln l$ & $\mathcal N(7.2, 1.2)$ & (2, 20) \\
    $\ln \Gamma$ & $\mathcal N(-2.3, 1.4)$ & (-10, 3) \\
    $\ln \sigma$ & $\mathcal N(-17, 5)$ & (-20, 0) \\
    $\ln P $ & Uniform / ACF-based & ($\ln 0.5, \ln 100$) \\
\end{tabular}
\end{center}
\end{table*}
\label{tab:priors}

\begin{figure*}
\begin{center}
% \includegraphics[width=6in, clip=true]{figures/compare_mcmc.pdf}
\includegraphics[width=6in, clip=true]{figures/comparison_acfprior_02_13.pdf}
\caption{The `true' rotation periods used to generate \naigrain\
simulated light curves vs the rotation periods measured using the GP
technique.
    Points are coloured by the peak-to-peak amplitude of the light curve, as
    defined in \citet{Aigrain2015}.
Since the posterior PDFs of rotation periods are often non-Gaussian,
    the points plotted here are maximum {\it a-posteriori} results.
The uncertainties are the 16th and 84th percentiles.
In many cases, the uncertainties are under-estimated.
The ACF-informed prior on rotation period used to generate these results is
    described in \textsection \ref{sec:GP_prior}.
    }
\label{fig:compare_mcmc_acfprior}
\end{center}
\end{figure*}

\begin{figure*}
\begin{center}
\includegraphics[width=6in, clip=true]{figures/comparison_noprior_02_13.pdf}
\caption{The `true' rotation periods used to generate \naigrain\
simulated light curves vs the rotation periods measured using the GP
technique with an uninformative prior.
    Points are coloured by the peak-to-peak amplitude of the light curve, as
    defined in \citet{Aigrain2015}.
Since the posterior PDFs of rotation periods are often non-Gaussian,
    the points plotted here are maximum {\it a-posteriori} results.
The uncertainties are the 16th and 84th percentiles.
In many cases, the uncertainties are under-estimated.
An uninformative prior, flat in the natural log of the rotation period between
    0.5 and 100 days was used to generate these results.
    }
\label{fig:compare_mcmc_noprior}
\end{center}
\end{figure*}


\begin{figure*}
\begin{center}
\includegraphics[width=6in, clip=true]{figures/corner_plot.pdf}
\caption{Marginal posterior PDFs of the QP GP model parameters, fit to the
    simulated light curve in figure \ref{fig:demo_lc}.
    The blue line in the period panel shows the injected period.
    This figure was made using \texttt{corner.py} \citep{Corner}.}
\label{fig:gp_posteriors}
\end{center}
\end{figure*}


\subsection{Comparison with literature methods}
\label{sec:comparison}
\subsubsection{ACF}
\label{sec:acf}

We measure an ACF-based period for each light curve,
following the method of
\citet{Mcquillan2013}.
After calculating the ACF for a light curve, we smooth it by convolving with a
Gaussian filter ($\sigma=9$\,d), and select the rotation period as the
lag-time of the highest peak in the ACF less than 100\,d.
This is not always the first peak---the second can be larger than the first if
two active regions are at or near opposite longitudes on the surface of
the star, producing a light curve with two dips per rotation period.
We restrict our search to periods less than 100\,d because instrumental
noise and light-curve systematics significantly distort \kepler\ light
curves on longer timescales, rendering longer periods realistically unrecoverable.
Figure \ref{fig:demo_lc} contains an example ACF of the light curve in
Figure \ref{fig:demo_acf}.

\begin{figure}
\begin{center}
\includegraphics[width=6in, clip=true]{figures/demo_ACF.pdf}
\caption[ACF of a simulated light curve.]
{An autocorrelation function of the simulated light curve shown in figure
\ref{fig:demo_lc}.
The vertical blue line shows the period measured using the ACF method (20.4
days) and the pink dashed line shows the period that was used to simulate the
light curve (\aigrainexampleperiod\ days).}
\label{fig:demo_acf}
\end{center}
\end{figure}

The ACF method has proven extremely useful for measuring rotation periods.
The catalogue of rotation periods of \Kepler\ stars provided in
\citet{Mcquillan2013} has been widely used by the community and has provided
ground-breaking results for stellar and exoplanetary science.
The method also performed well in the \citet{Aigrain2015}
recovery experiment,
producing a large number of accurate rotation period measurements
(see, e.g., their Figure 8).
Another advantage is its fast implementation speed.
However, because the ACF method is non-probabilistic, ACF-estimated
rotation period uncertainties are poorly defined---a clear
disadvantage.

We apply the ACF method to the sample of \naigrain\ simulated light
curves.
Figure \ref{fig:compare_acf} shows ACF-measured versus true rotation
periods, with the $2n$ and $\frac{1}{2}n$ harmonic lines as dashed lines.
The MRD of the ACF-recovered periods is \percentacfMAD\% (see table
\ref{tab:MADs} for a side-by-side comparison with the other methods).
The injected and recovered rotation periods generally agree well,
though with a few drastic over- or underestimates.
Additionally, of the points clustered close to the 1:1 line,
more fall slightly below than above it, \ie\
rotation periods are systematically underestimated.
This stems from peak position measurements in the ACF method.
ACFs of stellar light curves have similar functional forms to the QP kernel
function in Equation \ref{eq:QP}: periodic functions added to decaying
exponentials.
In such functions the peak positions can be shifted towards the left
(towards shorter periods), because the decaying exponential raises the left side of
each peak more than the right.
It is possible to model this effect; however, standard practice
simply measures the peak position without taking it into account.
We adopt this standard practice here to faithfully compare
our new method to that used in the literature.

\begin{figure*}
\begin{center}
\includegraphics[width=6in, clip=true]{figures/compare_acf.pdf}
\caption[ACF results.]
{The `true' rotation periods used to generate \naigrain\ simulated light
curves vs the rotation periods measured using the ACF technique.
    Points are coloured by the peak-to-peak amplitude of the light curve, as
    defined in \citet{Aigrain2015}.
    Several light curves have over-estimated rotation periods and some
    are drastically underestimated.}
\label{fig:compare_acf}
\end{center}
\end{figure*}

\subsubsection{LS periodogram}
\label{sec:ls}

For each simulated light curve, we compute a LS periodogram
\footnote{LS periodograms were calculated using the gatspy Python module:
\url{https://github.com/astroML/gatspy/tree/master/gatspy/periodic}.}
over a grid of 10,000 periods, evenly spaced in frequency,
between 1 and 100 days.
We adopt the period of the highest peak in the periodogram as the rotation
period.
Figure \ref{fig:pgram_compare} shows the resulting recovered rotation periods
as a function of true period.
The MRD of the periodogram-recovered periods is \percentpgramMAD\% (see table
\ref{tab:MADs} for a side-by-side comparison with the other methods).
This method drastically overestimates many rotation periods,
because long-term trends unrelated to rotation can cause excess power
at long periods.
% a problem further exacerbated by instrumental noise.

\begin{figure*}
\begin{center}
\includegraphics[width=6in, clip=true]{figures/compare_pgram.pdf}
\caption[LS periodogram results.]
{The `true' rotation periods used to generate \naigrain\ simulated light
curves vs the rotation periods measured using a LS periodogram technique.
    Points are coloured by the peak-to-peak amplitude of the light curve, as
    defined in \citet{Aigrain2015}.
In many cases a large peak at a long period was present in the periodogram,
    producing a significant over-estimate of the period.
    }
\label{fig:pgram_compare}
\end{center}
\end{figure*}

\begin{table*}
\begin{center}
    \caption{Median absolute (MAD) and median relative (MRD) deviations for
    the LS periodogram, ACF and GP period recovery methods.}
\begin{tabular}{lccc}
Method & MAD & MRD \\
    \hline
    LS periodogram & \pgramMAD\ days & \percentpgramMAD \% \\
    ACF & \acfMAD\ days & \percentacfMAD \% \\
    GP (uninformative prior) & \gpMADnp\ days & \percentgpMADnp \% \\
    GP (acf prior) & \gpMAD\ days & \percentgpMAD \% \\
\end{tabular}
\end{center}
\end{table*}
\label{tab:MADs}

\section{Real \kepler\ data}
\label{sec:kepler}

In order to test our rotation period inference method on real data,
we apply it to a set of \nkoimcq\ \Kepler\ Object of Interest (KOI)
host stars, that had rotation periods previously measured by
the ACF method by \citet{Mcquillan2013}.  We use the pipeline-corrected flux
(\texttt{pdcsap\_flux} column in the \Kepler\ light curve table), median-normalized
and unit-subtracted, and mask out all known transiting planet candidate signals.
As with the simulated light curves, we randomly subsample each
light curve by a factor of 30 and split it into segments of about 300 points
for the purposes of evaluating the likelihood.  We also follow the same MCMC
fitting procedure as with the simulated data, using the ACF-based prior as before.

Initially, we also use the same priors on the hyperparameters for the KOIs
as for the simulated light curves (Table \ref{tab:priors}).
However, we found that $\ln l$ and $\ln \Gamma$ tended toward slightly different
values than the simulations.  We also found that the allowed hyperparameter
range that we used for the simulations was too large for the KOI population,
as maybe $\sim$15\% of the fits tended toward corners in the hyperparameter
space, resulting in poor period measurements.  As a result, after this initial
test, we subsequently adjusted the priors and re-fit all the KOIs.
The final priors and parameter ranges that we used are in Table \ref{tab:koipriors}.

\begin{table*}
\begin{center}
\caption{Priors and bounds on the natural logarithms of the GP model parameters,
        for \Kepler\ light curves}
\begin{tabular}{lcc}
Parameter & Prior & Bounds\\
    \hline
    $\ln A$ & $\mathcal N(-13, 5.7)$ & (-20, 0) \\
    $\ln l$ & $\mathcal N(5.0, 1.2)$ & (2, 8) \\
    $\ln \Gamma$ & $\mathcal N(1.9, 1.4)$ & (0, 3) \\
    $\ln \sigma$ & $\mathcal N(-17, 5)$ & (-20, 0) \\
    $\ln P $ & ACF-based & ($\ln 0.5, \ln 100$) \\
\end{tabular}
\end{center}
\end{table*}
\label{tab:koipriors}

Figure \ref{fig:mcquillan} compares the periods inferred with the GP method to
the ACF-based periods from \citet{Mcquillan2013} for the \nkoimcq\ overlapping
KOIs.
This comparison shows generally very good agreement, with only a few
exceptions, demonstrating that this method works not only on simulated data,
but also on real data---with the caveat that for any particular data set, some
care is needed regarding the setting the priors and ranges for the GP
hyperparameters.
Notably, the GP method recovers periods
systematically slightly larger than \citet{Mcquillan2013}---a likely consequence
of correcting for the ACF peak measurement bias discussed in Section \ref{sec:acf}.

\begin{figure}
\begin{center}
\includegraphics[width=6in, clip=true]{figures/comparison_koi_02_03.pdf}
\caption[Comparison with McQuillan results.]
{A comparison of our GP rotation period measurements to those of
\citet{Mcquillan2013}.
The data points are coloured by the range of variability measured by
    \citet{Mcquillan2013}, defined as the interval between the 5th and 95th
    percentiles of normalised flux per period bin in millimagnitudes.}
\label{fig:mcquillan}
\end{center}
\end{figure}

\section{Discussion}
\label{sec:discussion}

The QP-GP inference method we present in this paper produces
more accurate rotation periods than the ACF or periodogram methods.  
Additionally, because it explores the posterior PDF with MCMC, 
our method produces probabilistically justified uncertainties.  
Unfortunately, these uncertainties still appear to be underestimated.
Of the rotation periods recovered from the simulated light curves, only 25\%
of the measured periods lie within 1$\sigma$ of the true period,
50\% lie within 2$\sigma$, and 66\% within 3$\sigma$.
The largest outlier is 114$\sigma$ away from the true value.
\tdmcomment{I assume this is for the ACF-based prior run? (should probably be mentioned)  
Are these numbers similar for the logflat prior?}
% In the majority of these cases the uncertainties are underestimated due to the
% multi-modal nature of the period posterior PDFs, which makes them difficult to
% sample. \tdmcomment{Is there any evidence of this?  Doesn't sound convincing to me.
% I would think the multi-modality would only be relevant if the posterior is
% closer to the 1/2x or 2x period, but this isn't most of them.}
We attribute these underestimated uncertainties to the model choice.
Although more appropriate than a sinusoid, the GP is still only an
 effective model.
A perfect physical model of the star would produce more representative
uncertainties.

The QP kernel function represents a simple effective model of a stellar
light curve.
It can describe a wide range of quasi-periodic behavior and is
relatively simple, with only a few hyperparameters.
Nevertheless, it is still a somewhat arbitrary choice.
Another valid choice would be a squared cosine function multiplied by a
squared exponential, used by \citet{Brewer2009} to model asteroseismic
pulsations,
\begin{equation}
k_{i,j} = A \exp \left(-\frac{(x_i - x_j)^2}{2l^2}\right)
\cos\left(\frac{\pi(x_i - x_j)}{P}\right).
\end{equation}
\label{eq:cos_kernel}
This function produces a positive semi-definite matrix and has the $P$
parameter of interest, but differs qualitatively from the QP function
by allowing negative covariances.
Is it realistic to allow negative covariances?
In practice, the ACFs of \Kepler\ light curves often go negative.
However, many stars have two active regions on opposite hemispheres that
produce two brightness dips per rotation, and it
may be difficult to model such stars with a kernel that forces
anti-correlation of points $\frac{1}{2}$ a period apart.
It would be very worthwhile to test this assumption and this alternative
kernel function (and others) in the future.
% If CPU time were not limited there may be some benefit to performing formal
% model comparison with different kernel functions.
% However, the evidence integral is an ambitious calculation for models with
% likelihood functions that take milliseconds to compute, let alone those
% involving GPs with light curves containing thousands of data points which can
% take minutes.
% The kernel we have chosen to use is only one of many 
% possibilities, and is
% already so flexible it does a fantastic job of reproducing more or less every
% stellar light curve we have used it to model.
% This in turn implies we should be cautious in interpreting the
% hyperparameters as physical quantities.
% Even if the period is more meaningful than the others, there are certain
% regimes (depending on the value of the other hyperparameters relative to the
% period) where the period is not physically meaningful.

Not all \kepler\ light curves show evidence of stellar rotation.
In some cases a star may have few or no active regions, be rotating
pole-on, or be rotating so slowly that the \kepler\ data detrending pipeline
removes any signal.
In other cases there may be another source of variability present in the light
curve, generating a false period detection.
These sources may be physical: \eg\ binary star interactions, intra-pixel
contamination from other astrophysical objects, pulsating variable stars,
asteroseismic oscillations in giants and even stellar activity cycles.
Identifying many of these astrophysical false positives 
falls outside the scope of this GP method 
(\eg\ applying colour cuts to avoid giant contamination);
however, for some, such as variable stars, they may have distinctive
hyperparameters (\eg\ long coherence timescales) that identify them.
Testing this may be an interesting follow-up study.
As well as astrophysical contamination, instrumental sources may contribute
to contaminating variability, \eg\ temperature variations or pointing shifts
of the \kepler\ spacecraft.
These are unlikely to be periodic and, again, may produce unusual combinations
of hyperparameters.
We also hope to test this in the future.

In addition, we are continuing to develop several other aspects of this GP method:
\begin{itemize}
\item{To perform model selection with different kernel functions. 
    We have not performed these tests yet due to the cost of calculating the
        fully marginalised likelihood with GPs.}
\item{To design and implement more physically motivated priors.
We have only explored the physical interpretation of one parameter in our
        kernel function ($P$), but others may warrant physically motivated
        priors.
For example, take the harmonic complexity parameter, $\Gamma$.
        For large values of $l$ ($l\gg P$), $\Gamma$ relates to the
        typical number of stationary/non-stationary points of inflection per
        period.
        % found in functions drawn from the corresponding GP.
        This behaviour will be independent of the exact value of $l$.
        For `intermediate' values of $l$ ($l\gtrsim P$) this still holds
        but the behaviour will be sensitive to the value of $l$.
        Small values of $l$, \eg\ $l<P$, are probably not relevant -- see
        below.
        % Depending on the processes giving rise to observed stellar signals,
        % you may not expect to require certain types of functions to model
        % observed signals.
        In typical light curves, observing more than perhaps three or four
        turning points per period in the disk-integrated photometric signal
        (of course, $2$ turning points is the minimum
        for periodic functions) is highly unlikely.
        So there is likely to be some maximum $\Gamma$, related to the maximum
        number of turning points typically observed in stellar light curves.
        % Such simulations, crude though they may be, could be used to come up
        % with physically-motivated priors for $\Gamma$.
For the evolution time-scale, $l$, it is sensible to enforce $l>P$, such that
        functions that have clear periodicity.
        Intuitively, if $l<P$ the functions may evolve significantly over time
        scales shorter than one period, so a function $f(t+P)$ would look
        signficantly different to $f(t)$, in which case any claims of
        periodicity become dubious.}
        % A more rigorous criterion (which also
        % takes into account dependencies on $\Gamma$) is presented below.
        % Provided
        % \begin{equation}
        % \frac{{{P^2}}}{{{l^2}{\Gamma ^2}}} < \frac{{4\pi }}{3},
        % \end{equation}
        % the QP covariance function will have at least one inflection point
        % other than at $|x_i-x_j|=0$, so $\frac{{{P^2}}}{{{l^2}{\Gamma ^2}}} =
        % \frac{{4\pi }}{3}$
        % corresponds to a critical point at which secondary maxima disappear,
        % and the behaviour of the kernel becomes dominated by the SE term
        % rather than the periodic term.
        % For $\Gamma\sim\mathcal{O}(1)$, this translates into $l\gtrsim P/2$.
        % In practice, satisfying this criterion doesn't absolutely guarantee
        % that function draws will have clear evidence of periodicity, but it
        % does provide a quantitative boundary between obviously periodic
        % behaviour and not-so-obviously periodic behaviour.
        % It also provides some justification for a more stringent criterion
        % such as $l>P$, or $l>2P$ (ensure that functions
        % evolve significantly over time-scales longer than 2 periods).}
% However, the other parameters may also be related to some physical processes.
% For example, the overall timescale for covariance fall-off, $l$ may be
% related to spot lifetimes.
% A star with long spot lifetimes will show little variation in the overall
% shape and amplitude of its light curve between rotations and $l$ will be
% large.
% In contrast, the light curve of a star with short spot lifetimes may display
% non-repeating patterns and amplitudes that vary rapidly between rotations.
% In this case $l$ will be small.
% The $\Gamma$ parameter is related to the number of zero crossings within one
% rotation period: when $\Gamma$ is small there are many zero crossings and
% vice versa.
% Since the number of zero crossings per rotation period is related to the
% number of active regions on the surface of the star, this parameter may also
% be of physical interest.
% In addition, instead of interpreting the parameters of the QP kernel function
% used here, it may be possible to design an entirely new kernel function, based
% on the physical processes that drive the light curve variability.
% This idea is being explored by another member of my research group.}
% \item \vrcomment{[VR: for the two points above, you might want to reword
%     slightly depending on the extent to which you want to try to (i)
%         incorporate my suggestions re prior constraints etc.\ etc.\ vs.\ (ii)
%         deferring this to the future paper, but at least alluding to the
%         possibilities and results. For example I think one needn't try to
%         `design' a physically-motivated kernel -- rather it seems not too
%         difficult just to come up with physically-motivated priors for the QP
%         kernel parameters. I've tried to design a physically motivated kernel
%         to capture differential rotation but as noted earlier, that came with
%         its own slew of degeneracies and other problems...more on this another
%         time.]}
\item{To build in a noise model for \kepler\ data.
Because it is a {\it generative} model of the data,
the GP method models the rotation period at the same time as systematic
noise features.
One can then marginalise over the parameters of the noise model.
This approach would be extremely advantageous for \kepler\ data since
long-term trends are often removed by the \kepler\ detrending pipeline.
Marginalising over the noise model at the same time as inferring the
parameters of interest will insure that the periodic signal is preserved. 
\tdmcomment{Is this bullet point obselete, given that we have already successfully
marginalized over \kepler\ noise to recover periods? Or is there
something else we wanted to say here?}}
\end{itemize}

\section{Conclusion}

We have attempted to recover the rotation periods of \naigrain\ simulated
\kepler-like light curves for solid-body rotators \citep{Aigrain2015} using
three different methods: a new Gaussian process method, 
an authocorrelation function method, and a Lomb-Scargle periodogram method.
We demonstrate that the GP method produces the most accurate rotation periods
of the three techniques, providing a large improvement over the LS periodogram
method and a moderate one over the ACF method.
We also find that the standard version of the ACF method, most commonly
implemented in the literature, often slightly under-predicts the rotation
period due to a subtle feature of its peak detection algorithm.
In addition, we measure the rotation periods of \nkoimcq\ \kepler\ objects of
interest using the GP method, and find that these results compare well
to those measured previously by \citet{Mcquillan2013}.
The good agreement between these two sets of results demonstrates that the GP
method works well on real \kepler\ data.

Unlike the ACF and L-S periodogram methods, the GP method provides posterior
PDF samples which can be used to estimate rotation periods uncertainties.
Although an improvement on competing methods, these uncertainties are still
underestimated in many cases because the GP model is an approximate, effective
model. \tdm{Can we make a statement something like ``we believe the method underestimates
uncertainties by about a factor of three'' or something like that?  This
matches with 66\% of the fits being within ``3$\sigma''.
Might be good to say something like this, both in the discussion and here.}
% in many cases the posterior PDFs of the parameters are
% multi-modal.
% If it were possible to run an MCMC sampler for an infinite number of steps,
% the entire posterior PDF would eventually be sampled, and a more accurate
% uncertainty would result.
% Unfortunately, due to realistic computational resources and the limitations of
% MCMC sampling techniques for multi-modal posteriors, we are not always able to
% map out the entire posterior PDFs.
% \tdmcomment{I don't think this is the problem;
% I think the fits that we judge to have converged have actually converged.}
In addition, although the GP model used here is clearly a good one, it is
still only an effective model, not an accurate physical model.
It can only capture the posterior PDF of the periodic component of the
covariance matrix, not the actual rotation period of a physical star.

The main aim of this work was to develop a probabilistic rotation period
inference method.
Probabilistic periods are necessary for hierarchical Bayesian inference,
particularly when performing population analysis.
This new GP method is capable of generating a catalogue of probabilistic
rotation periods which could, for example, reveal the period distribution
(and therefore potentially the age distribution) of stars in the Milky Way.
There is still room for improvement since the uncertainty estimates are not
yet truly representative and it is still only an `effective' model, not a
physical one.
However, it is probabilistic and provides more accurate periods than
alternative methods.
We therefore argue that the GP method is the best method for rotation period
inference currently available.

% acknowledgements
This research was funded by the Simons Foundation and the Leverhulme Trust.
TDM is supported by NASA grant NNX14AE11G, and acknowledges the
hospitality of the Institute for Advanced Study,
where part of this work was completed.
VR thanks Merton College and the National Research Foundation of South Africa
for financial support.
Some of the data presented in this paper were obtained from the Mikulski
Archive for Space Telescopes (MAST).
STScI is operated by the Association of Universities for Research in
Astronomy, Inc., under NASA contract NAS5-26555.
Support for MAST for non-HST data is provided by the NASA Office of Space
Science via grant NNX09AF08G and by other grants and contracts.
This paper includes data collected by the Kepler mission. Funding for the
Kepler mission is provided by the NASA Science Mission directorate.

% \appendix

% \section{An ACF-informed Prior for Rotation Period}

% The autocorrelation function (ACF) has proven to be very useful
% for measuring the rotation periods of stars from time-series
% photometry.
% However, as discussed in the body of this paper, the
% ACF-based method has several shortcomings,
% most notably the inability to deliver uncertainties, but also
% the necessity of several heuristic choices,
% such as a timescale on which to smooth the ACF,
% how to define a peak, whether the first or second peak
% gets selected, and what constitutes a secure detection.
% While this paper presents a new MCMC-based method
% that avoids these shortcomings,
% it seems prudent to still use information available from the ACF.
% This Appendix describes how we use the ACF to define a prior on period.

% Our strategy here is not to attempt to decide which single
% peak in the ACF best represents the true rotation period,
% but rather to identify several \emph{candidate} periods and define
% a weighting scheme in order to create a non-commital, though useful,
% multimodal prior.  While this procedure does not avoid heuristic choices,
% the fact that its goal is simply to create a \emph{prior} for a
% probabilitistically justified inference scheme rather than
% to identify a single correct period softens the potential impact
% of these choices.

% As another innovation beyond what ACF methods in the literature have
% presented, we also bandpass filter the light curves (using a 5th order
% Butterworth filter, as implemented in \texttt{SciPy}) before calculating the
% autocorrelation.  This suppresses power on timescales shorter than a chosen
% minimum period $P_{\rm min}$ and longer than a chosen maximum $P_{\rm max}$,
% producing a cleaner autocorrelation signal than an unfiltered light curve.

% We use the following procedure to construct a prior for rotation period
% given a light curve:
% \begin{enumerate}
% \item{For each value of $P_i$, where $i = \{1, 3, 5, 10, 30, 50, 100\}$\,d,
% we apply a bandpass filter to the light curve using $P_{\rm min}=0.1$\,d
% and $P_{\rm max} = P_i$.  We then calculate the ACF of the filtered
% light curve out to a maximum lag of $2P_i$ and smooth it with a boxcar
% filter of width $P_i/10$.}

% \item{We identify the time lag corresponding to the
% first peak of each of these ACFs, as well as the first peak's
% trough-to-peak height, creating a set of candidate periods
% $T_i$ and heights $h_i$.}

% \item{We assign a quality metric $Q_i$ to each of these candidate
% periods, as follows.  First, we model the ACF as a
% damped oscillator with fixed period $T_i$:
% \begin{equation}
% y = A e^{-t/\tau} \cos{\frac{2\pi t}{T_i} },
% \end{equation}
% where and $t$ is the lag time,
% and find the best-fitting parameters $A_i$ and $\tau_i$ by a non-linear
% least squares minimization procedure.  We then define the
% following heuristic quality metric:
% \begin{equation}
% \label{eq:quality}
% Q_i = \left(\frac{\tau_i}{T_i}\right) \left(\frac{N_i h_i}{R_i}\right),
% \end{equation}
% where $h_i$ is the height of the ACF peak at $T_i$,
% $N_i$ is the length of the lag vector in the ACF (directly proportional
% to the maximum allowed period $P_i$),
% and $R_i$ is the sum of squared residuals between the
% damped oscillator model and the actual ACF data.  The idea behind this
% quality metric to give a candidate period a higher score if

%     \begin{enumerate}[(a)]

%     \item{it has many regular sinusoidal peaks, such that the decay
%         time $\tau_i$ is long compared to the oscillation period $T_i$,}
%     \item{the ACF peak height is high, and}
%     \item{the damped oscillator model is a good fit (in a $\chi^2$
%       sense) to the ACF, with extra bonus for being
%       a good fit over more points (larger $N_i$, or longer $P_i$).}

%     \end{enumerate}
% }

% \item{Given this set of candidate periods $T_i$ and quality metrics $Q_i$,
% we finally construct a multimodal prior on the $P$ parameter of the GP
% model as a weighted mixture of Gaussians:
% \begin{equation}
% \label{eq:mixture}
% p(\ln P) = \frac {\displaystyle \sum_i Q_i \left(0.9\mathcal N(\ln T_i, 0.2) +
%                                           0.05\mathcal N(\ln (T_i/2), 0.2) +
%                                           0.05\mathcal N(\ln (2 T_i), 0.2) \right)}
%                 {\sum_i Q_i}.
% \end{equation}
% That is, in addition to taking the candidate periods themselves as mixture
% components, we also mix in twice and half each candidate period at a lower level,
% which compensates for the possibility that the first peak in the ACF may actually
% represent half or twice the actual rotation period.  The period width of 0.2 in
% log space (corresponding to roughly 20\% uncertainty) is again a heuristic choice,
% balancing a healthy specificity with the desire to not have the results of
% the inference overly determined by the ACF prior.
% }

% Incidentally, while we use the procedure described here to create a prior on
% $P$ which we use while inferring the parameters of the quasi-periodic GP model,
% this same procedure may also be used in the service of a
% rotation-period estimating procedure all on its own, perhaps being even more
% robust and accurate than the traditional ACF method.  We leave exploration of
% this possibility to future work.



\bibliographystyle{plainnat}
\bibliography{GProtation}
\end{document}
