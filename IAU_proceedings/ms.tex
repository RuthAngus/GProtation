% This paper is part of the k2-paper project.
% Copyright 2015 Dan Foreman-Mackey (NYU) and the co-authors listed below.
%
%  RULES OF THE GAME
%
%  * 80 characters
%  * line breaks at the ends of sentences
%  * eqnarrys ONLY
%  * ``light curve'' not ``light-curve'' or ``lightcurve''
%  * Do not put in any comments that might get tweeted by @OverheardOnAph
%    (or maybe do put in a few....)
%  * ``percent'' (not \%) is a unit, as is ppm, so 5~percent.
%  * that is all.
%

\documentclass[11pt,preprint]{aastex}

\pdfoutput=1

\usepackage{color,hyperref}
\definecolor{linkcolor}{rgb}{0,0,0.5}
\hypersetup{colorlinks=true,linkcolor=linkcolor,citecolor=linkcolor,
            filecolor=linkcolor,urlcolor=linkcolor}
\usepackage{url}
\usepackage{amssymb,amsmath}
\usepackage{subfigure}
\usepackage{booktabs}

\usepackage{natbib}
\bibliographystyle{apj}

\newcommand{\project}[1]{\textsl{#1}} % hogg say
\newcommand{\kepler}{\project{Kepler}}
\newcommand{\KT}{\project{K2}}
\newcommand{\tess}{\project{TESS}}
\newcommand{\jwst}{\project{JWST}}
\newcommand{\license}{MIT License}
\newcommand{\paper}{\textsl{Article}}
\newcommand{\foreign}[1]{\emph{#1}}
\newcommand{\etal}{\foreign{et\,al.}}
\newcommand{\etc}{\foreign{etc.}}
\definecolor{mygreen}{rgb}{0, 0.50196, 0}
\newcommand{\response}[1]{{\color{mygreen} {\bf #1}}}
\newcommand{\figlabel}[1]{\label{fig:#1}}

\begin{document}

\title{%
	Stellar rotation periods inference with Gaussian processes
}

\newcommand{\oxford}{1}
\newcommand{\UW}{2}
\author{%
    Ruth~Angus\altaffilmark{1,\oxford},
    Suzanne~Aigrain\altaffilmark{\oxford}
    Daniel~Foreman-Mackey\altaffilmark{\UW},
}
\altaffiltext{1}         {To whom correspondence should be addressed:
                          \url{ruth.angus@astro.ox.ac.uk}}
\altaffiltext{\oxford}      {Subdepartment of Astrophysics, Denys-Wilkinson
	building, Keble road, Oxford, OX1 3RH}

\begin{abstract}

The light curves of spotted, rotating stars are often non-sinusoidal and
Quasi-Periodic (QP) and a sinusoid is therefore not a representative generative
model for stellar light curves.
Sine-fitting periodograms only provide the period of the best fitting
sinusoid, not the period of the best fitting rotating star model.
Ideally, a physical model of the stellar surface would be conditioned on the
data, however the parameters of such physical models can be highly degenerate.
Instead, we use an appropriate {\it effective} model: a Gaussian Process (GP)
with a QP covariance kernel function.
By modelling the covariance matrix of the light curve with a QP kernel
function, we remain agnostic about model choice, whilst sampling directly from
the posterior probability distribution function of the periodic parameter and
marginalising over the other kernel hyperparameters.
We simulated light curves with a range of rotation periods and spot lifetimes
and attempted to recover the rotation periods using three methods: our GP
method, a sine-fitting periodogram method and an AutoCorrelation Function (ACF)
method.
The GP method produces rotation period measurements that are more precise than
the periodogram and both more accurate and precise than the ACF method.

\end{abstract}

\section{Introduction}

\section{Method}

\section{Results}

\section{Conclusion}

\begin{figure}[p]
\begin{center}
\includegraphics{compare.png}
\end{center}
\caption{%
Measured vs true rotation periods for 300 simulations of light curves from
spotted, rotating stars.
Three different methods were tested: the ACF method, a Lomb-Scargle
periodogram (sine-fitting) method and my new GP method.
The GP method measures the most precise and accurate rotation periods
and is expected to perform even better on real data.
\figlabel{rotation}}
\end{figure}

{\it Facilities:} \facility{Kepler}

% \bibliography{GP}

\end{document}
